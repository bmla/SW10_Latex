% !TEX root = ../report.tex
\addcontentsline{toc}{section}{Summary}
\begin{dummy}

	\drop=0.1\textheight
	\centering
	\vspace*{\baselineskip}
	\rule{\textwidth}{2pt}\vspace*{-\baselineskip}\vspace*{2pt}
	\rule{\textwidth}{0.4pt}\\[\baselineskip]
	{\LARGE Summary}\\[0.2\baselineskip]
	\rule{\textwidth}{0.4pt}\vspace*{-\baselineskip}\vspace{3.2pt}
	\rule{\textwidth}{2pt}\\[\baselineskip]
	\scshape
	{} \par

\begin{summary}
In this master thesis, two experimental studies were created with the purpose of comparing eight different cross-device interaction techniques allowing users to push and pull data to and from a large display using a smartphone.
The studies complements existing research in the area of cross-device data transfer between handheld devices and large displays by providing a quantitative comparison of different techniques which combines finger pointing, pointing with phones, and one- and two-handed interactions.

To perform the studies and collect data, a system was developed using a Microsoft Kinect, a large display, and a smartphone.
The studies were conducted in the usability lab and ran for four weeks with a total of 84 people participating. 

This thesis builds on knowledge gained in the authors' 9th semester pre-specialisation, and further development of the system in the 10th semester specialisation has resulted in a scientific paper presenting the studies and the results.
The paper concludes that four techniques (\pull and \push versions of the \swipe and \throw techniques) were more successful than the other four techniques (\pull and \push versions of the \grab and \tilt techniques).
Because the best techniques, \swipe and \throw, have different attributes it might be useful for interaction designers to know which techniques to consider when creating systems that use this kind of cross-device interaction. 
Since the techniques have different defining attributes they can be used in different situations and contexts.
\end{summary}
	\vspace*{2\baselineskip}
		{\scshape 2016} \\
		{\large AALBORG UNIVERSITY}\par
	
\end{dummy}