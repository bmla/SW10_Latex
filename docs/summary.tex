% !TEX root = ../report.tex
\addcontentsline{toc}{section}{Summary}
\begin{dummy}

	\drop=0.1\textheight
	\centering
	\vspace*{\baselineskip}
	\rule{\textwidth}{2pt}\vspace*{-\baselineskip}\vspace*{2pt}
	\rule{\textwidth}{0.4pt}\\[\baselineskip]
	{\LARGE Summary}\\[0.2\baselineskip]
	\rule{\textwidth}{0.4pt}\vspace*{-\baselineskip}\vspace{3.2pt}
	\rule{\textwidth}{2pt}\\[\baselineskip]
	\scshape
	{} \par
	\vspace*{2\baselineskip}
	%Edited by \\[\baselineskip]
	\vspace*{2\baselineskip}

\begin{summary}
In this master thesis, two experimental studies were created with the purpose of comparing eight different interaction techniques allowing users to push and pull data to and from a large display using a smartphone.

To perform the studies and collect data, a system was developed using a Microsoft Kinect, a large display, and a smartphone.
The studies were conducted in the usability lab and ran for four weeks with a total of 84 people participating. 

This thesis builds on knowledge gained in the authors' 9th semester pre-specialisation and the further development of the system in the 10th semester specialisation has resulted in a scientific paper presenting the studies and the results.
The paper concludes that two techniques were more successful than the other eight techniques.
Because the two best techniques have different attributes it might be useful for interaction designers to know which techniques to consider when creating systems that use this kind of cross-device interaction, and deciding which technique should be used instead of the other might depend on the situation.
\end{summary}
	\vspace*{2\baselineskip}
		{\scshape 2016} \\
		{\large AALBORG UNIVERSITY}\par
	
\end{dummy}