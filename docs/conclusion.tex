% !TEX root = ../report.tex
\section*{Concluding Remarks} \label{sec:conclusion}
\addcontentsline{toc}{section}{Concluding Remarks}

This thesis deals with the theme of cross-device interaction between mobile devices and large displays. 
We approached this theme by finding and implementing 8 different interaction techniques and then performing a comparative study between them.

One of the ideas behind this thesis was to see if we could find any attributes to these techniques that made them successful in regards to accuracy and efficiency.
We initially believed that the some of the important attributes to the success of a technique would be whether or not the phone was in motion during the technique and the amount of hands used to perform the given technique.

Our results did show that there might be some association between the amount of hands and the accuracy and efficiency of each technique. 
There seems to be some indication that one handed techniques are faster to perform that two handed, but that two handed techniques are more accurate.
These are only indications though and in order to more conclusively say that this is indeed the case, more research must be conducted with these specific attributes in mind.  

A much more important attribute though came up and that was the ability to hold the cursor still while performing the technique. 
The two most successful techniques, \emph{Swipe} and \emph{Throw} both had that in common. 
Users where capable of keeping the cursor still while activating and performing the technique.
The other two techniques, \emph{Tilt} and \emph{Grab}, both movements on the cursor pointing hand, causing the cursor to move during activation instead of keeping it stable. 

In the future, we would like to extend our research by examining more closely the relationship between amount of hands and the efficiency and accuracy of each technique. 
This could be done by implementing different techniques were the focus is much more on the amount of hands and the role of each hand while performing the given technique.
Having techniques were the sole role of one of the hands is for aiming, like \emph{Throw}, and others were each hand has some gesture it has to perform in order to activate the technique, such as \emph{Grab}.
A research like this could lead to a much more clear understanding of how the amount of hands affects the performance of a interaction technique.