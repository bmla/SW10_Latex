% !TEX root = ../report.tex
\section*{Introduction}\label{sec:introduction}
\addcontentsline{toc}{section}{Introduction}
Our initial motivation for this and last semester's project was our interest in Cross-Device Interaction, Natural User Interfaces (NUI), and public spaces.
One previous masters project that inspired us was a collaborative touch based wall, where users could come up and create music together by touching the wall at different places\footnote{Damgaard, Madsen and Sørensen. 2011. \textit{Experiencing music together: Control and Identity} http://goo.gl/yp9J5t}.

We started out by narrowing the subject and coming up with the current theme: cross-device interaction techniques between mobile devices and large displays. 
We then decided to work on a project were the main idea was to compare different cross-device interaction techniques that would be used to exchange data between mobile devices and large displays. 

This is a relatively new field of study and as such, provides us ample opportunity to contribute meaningful and relevant information to its pool of knowledge. 
We are excited to be part of this exploration of cross-device, natural user interaction techniques. 

The main product of this semester, the paper delivered in the report, presents a research experiment were we explore and compare four different cross-device interaction techniques for pushing data to a large display and four cross-device interaction techniques for pulling data from a large display. 
These techniques are all techniques that have been used before in research and prototypes, so that we could study something meaningful and relevant.
These techniques are measured in terms of their accuracy and precision, in comparison to each other.
This is done by creating an experiment in which users are asked to perform the each technique and hit targets that are displayed on a large display.
We then present and discuss the given results. 