\newcommand{\grab}{\textit{Grab}\xspace}
\newcommand{\swipe}{\textit{Swipe}\xspace}
\newcommand{\throw}{\textit{Throw}\xspace}
\newcommand{\tilt}{\textit{Tilt}\xspace}

\newcommand{\push}{\textit{Push}\xspace}
\newcommand{\pull}{\textit{Pull}\xspace}

\newcommand{\effectiveness}{\textit{Effectiveness}\xspace}
\newcommand{\direction}{\textit{Direction}\xspace}
\newcommand{\technique}{\textit{Technique}\xspace}
\newcommand{\targetsize}{\textit{Target Size}\xspace}

\newcommand{\alltechniques}{ \grab, \swipe, \throw and \tilt}
\newcommand{\target}{\textit{Target-Study}\xspace}
\newcommand{\accuracy}{\textit{Accuracy-Study}\xspace}
\newcommand{\studyone}{\textit{Study One}}
\newcommand{\studytwo}{\textit{Study Two}}
\newcommand{\ts}{\textsuperscript}

\newcommand\T{\rule{0pt}{2.6ex}}       % Top strut
\newcommand\B{\rule[-1.2ex]{0pt}{0pt}} % Bottom strut

\newcommand\todoin[2][]{\todo[inline, caption={2do}, #1]{
\begin{minipage}{\textwidth-4pt}#2\end{minipage}}}

\newcommand\highlight[2][]{\todo[inline, backgroundcolor=yellow, bordercolor=none, caption={2do}, #1]{
\begin{minipage}{\textwidth-4pt}#2\end{minipage}}}


\def\startCirc#1{\tikz[remember picture,overlay]\path node[inner sep=0, anchor=south] (st) {\textbf{#1}} coordinate (start) at (st.center);}%
\def\endCirc#1{\tikz[remember picture,overlay]\path node[inner sep=0, anchor=south] (en) {\textbf{#1}} coordinate (end) at (en.center);%
 	\begin{tikzpicture}[overlay, remember picture]%
 	\path (start);%
 	\pgfgetlastxy{\startx}{\starty}%
 	\path (end);%
 	\pgfgetlastxy{\endx}{\endy}%
 	\pgfmathsetlengthmacro{\xdiff}{\endx-\startx}%
 	\pgfmathsetlengthmacro{\ydiff}{\endy-\starty}%
 	\pgfmathtruncatemacro{\xdifft}{\xdiff}%
 	\pgfmathsetmacro{\xdiffFixed}{ifthenelse(equal(\xdifft,0),1,\xdiff)}%
 	\pgfmathsetmacro{\angle}{ifthenelse(equal(\xdiffFixed,1),90,atan(\ydiff/\xdiffFixed))}%
 	\pgfmathsetlengthmacro{\xydiff}{sqrt(abs(\xdiff^2) + abs(\ydiff^2))}%
 	\path node[draw,rectangle, rounded corners=4mm, solid, rotate=\angle, minimum width=\xydiff+8ex, minimum height=5ex, fill=gray, opacity=0.3] at ($(start)!.5!(end)$) {};%
 	\end{tikzpicture}%
}
