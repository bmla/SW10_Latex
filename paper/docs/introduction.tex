% !TEX root = ../paper.tex
\section{Introduction} \label{sec:introduction}
Displays are generally getting larger and larger and today it is possible to buy larger displays than it was only a few years ago.
Interacting with systems that use large displays could perhaps benefit from using mid-air pointing making it possible to use hands to navigate around the screen.
If we introduce a smartphone into this mid-air pointing and large display mix, we could interact with the large display using mid-air pointing and transfer information to the display such as movies, videos, pictures and other media from a smartphone using a technique.
Different kinds of techniques have been used in the literature and we have not been able to find research that compared \emph{mid-air techniques for interacting with large displays using smartphones}.

Mid-air interactions are being used more and more for different applications e.g. gaming and virtual and augmented reality.
Consumer depth cameras, like the more popular Kinect, have been around for years and allows its users to control a game using mid-air gestures or researchers to create new interfaces for applications and systems that utilizes e.g. pointing in mid-air.
Interaction between large displays and handheld devices has been studied previously but to our knowledge there has not been empirical studies comparing techniques for transferring information between large displays and smartphones using mid-air pointing as an interface for cursor control.
By comparing such techniques on success hit rate, time, and accuracy, we might be able to say which technique is the best and maybe if some techniques are more precise than others and if some are faster than others.

In our empirical study we compare and analyze the data collected from an experiment on 8 techniques, 4 techniques for a push direction (smartphone to display) and 4 techniques for a pull direction (display to smartphone).
We conducted two experiments in a controlled environment with 51 and 33 participants respectively.
In this paper we present an empirical study on mid-air techniques for interaction between a large display and a smartphone.