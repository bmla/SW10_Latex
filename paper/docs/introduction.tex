% !TEX root = ../paper.tex
\section{Introduction} \label{sec:introduction}
In line with technological advances, single screen digital displays, for both domestic and public use, are available in increasingly larger sizes than just a few years ago.
It has been shown that interacting with systems that use these large displays benefit from using mid-air pointing from a distance, making it possible to use hand gestures to navigate around the screen \cite{Jakobsen:2015}.
Mid-air interactions are being increasingly used for different applications, e.g., in gaming or virtual and augmented reality.
Inexpensive consumer depth cameras, like the popular Microsoft Kinect, have been around for years and allow users to control a game using mid-air gestures.
The Kinect has also been popular with researchers who have used them to explore new interfaces for applications and new human computer interaction possibilities, e.g., mid-air pointing or bodily movement to communicate with the system.

If we introduce a smartphone into this mid-air pointing and large display mix, we enable the transfer of information to and from the display, e.g., text, videos, images and other media, between the smartphone and the system.
But how do we decide which interaction techniques people should use with their mobile devices to make the transfer happen? Different kinds of mid-air techniques have been studied in different situations in the literature but we have not found any existing empirical research that compares alternative mid-air techniques using smartphones for two-way interaction with large displays.
An understanding of how different techniques compare with each other in terms of effectiveness, efficiency and accuracy could help interaction designers make decisions about which interactions to implement in their systems.
Interactions to support users pushing or pulling information between the system and their mobile devices could be designed based on which of these attributes is most important in a particular application context.
Another important aspect for interaction designers could be the “naturalness” or learnability of the alternative mid-air techniques, which might also influence their choice of technique, for example, in a walk up and use scenario in a public display.

To contribute to current knowledge on these issues, we have conducted an empirical study that compared and analyzed the data collected from an experiment on 8 different interaction techniques, that is, 4 techniques for push (from smartphone to display) and 4 techniques for pull (from display to smartphone).
By comparing these techniques on parameters of hit rate, time taken, and distance from target, with two studies in a controlled environment, we are able to demonstrate which techniques are more precise than others and which are faster to complete in a laboratory setting.

In the first study, we collect data to compare time and hit rate for each technique.
In the second study, we collected data to compare a technique's accuracy.

In this paper we report on our findings from both our experiments, showing that a swiping technique is the most effective and accurate one for exchanging data between a large display and a smart phone, compared to the other three. 
We also show that a grabbing and a throwing technique closely follow in regards to an efficient and accurate exchange of data between the two aforementioned devices. 

We also show that out of the four techniques we implemented, swipe was the most efficient, followed by a throwing technique. 
The grabbing technique that we implement is also shown to be the less efficient technique, because it is the most complex and is comprised of the most steps needed in order to perform it. 