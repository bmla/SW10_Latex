% !TEX root = ../paper.tex
\section{Experiment} \label{sec:experiment}
In order to be able to compare the different techniques to each other we conducted an experiment in which each participant would perform the techniques in a controlled environment.
Before a participant started the experiment, the general purpose of the study would be explained to the participant.
The 8 techniques were presented one technique after another and before a technique would start, a demonstration video of that technique were shown to the participants on the large display.
After two iterations of showing the video of the large screen the video runs in a loop on the smaller display.
Screenshots of the Push Throw technique demo is shown in \Cref{fig:demovideo}.

\begin{figure}[H]
\subfloat[]{\includegraphics[width = 0.5\columnwidth]{images/demovideo1.pdf}\label{fig:demovideA}}
\subfloat[]{\includegraphics[width = 0.5\columnwidth]{images/demovideo2.pdf}\label{fig:demovideB}}\\
\subfloat[]{\includegraphics[width = 0.5\columnwidth]{images/demovideo3.pdf}\label{fig:demovideC}}
\subfloat[]{\includegraphics[width = 0.5\columnwidth]{images/demovideo4.pdf}\label{fig:demovideD}}
\caption{The screen at different times during the demonstration video. The video was shown to the participant before each technique test starts. \protect\subref{fig:demovideA} Presenting the technique with the direction and the name. \protect\subref{fig:demovideB}, \protect\subref{fig:demovideC}, \protect\subref{fig:demovideD} The video pauses and the participant will be able to read the instructions on the screen.}
\label{fig:demovideo}
\end{figure}

The experiment was conducted in the usability lab at Aalborg University and in the lab we had setup a large 65 inch screen and a smaller 42 inch screen. 
A Kinect v2 was mounted below the large display.
We marked a spot on the floor with a cross where participants were asked to stand.