% !TEX root = ../paper.tex
%http://sphweb.bumc.bu.edu/otlt/MPH-Modules/BS/BS704_HypothesisTesting-ANOVA/BS704_HypothesisTesting-Anova_print.html
\section{Results}\label{sec:results}

Some data points needed to be removed from the experiments, in order to gain a clearer understanding of the data we just gathered. 


From the \target, we started with a total of 7344 attempts.
176 were removed because of system errors, were the system wrongly activated a technique attempt even though the user did not intend to do so.
Another 232 attempts were removed as outliers using the Outlier Labeling method described by Hoaglin and Iglewicz in Resistant Rules for Outlier Labeling \cite{Hoaglin:1987}.
This gave us a total of 6936 attempts for the \target.

From the \accuracy, we started with a total of 4752 attempts.
111 attempts were removed to system errors. 
Finally 130 attempts were removed with the same Outlier Labeling method used above.
This gave us a total of 4511 attempts for the \accuracy.

We then split up the data into two different data sets, the \push and \pull, in order to look at them independently. 

\Cref{tab:numberOfAttempts} shows the final number of attempts each technique had in the two different experiments

\begin{table}[H]
	\centering
	\textbf{Number of attempts}\\[4pt]
	\begin{adjustbox}{width=\columnwidth}
		\subfloat[\target]{
			\begin{tabular}{|c|c|c|c|c|}
				\hline
				\rowcolor[HTML]{9B9B9B} 
				& \textbf{Grab} & \textbf{Swipe} & \textbf{Throw} & \textbf{Tilt} \\ \hline
				Push & 830 & 896 & 889 & 862 \\ \hline
				Pull & 797 & 893 & 896 & 873 \\ \hline
			\end{tabular}
		}
		\subfloat[\accuracy] {
			\begin{tabular}{|c|c|c|c|c|}
				\hline
				\rowcolor[HTML]{9B9B9B} 
				& \textbf{Grab} & \textbf{Swipe} & \textbf{Throw} & \textbf{Tilt} \\ \hline
				Push & 551 & 583 & 566 & 575 \\ \hline
				Pull & 534 & 564 & 568 & 570 \\ \hline
			\end{tabular}
		}
	\end{adjustbox}
	\caption{Number of attempts for each technique in each experiment}
	\label{tab:numberOfAttempts}
\end{table}


\subsection{Success rate}
The results presented here will be based on the data collected during the \target. 
Here we will be presenting results relating to whether or not the user hit the target, which we will be referring to as effectiveness when discussing the results. 

To see whether or not each technique had an effect on the effectiveness of each attempt, we performed a Pearsons Chi-Square test on both data sets. 
For the \push techniques, $X(3)=121.950$, $p<0.000$, and for the \pull techniques we got $X(3)=438.473$, $p<0.000$. 
This means that both \push and \pull techniques had a significant effect on the effectiveness of each attempt. 
\Cref{tab:successRate} shows the success rate for each of the techniques. 

\begin{table}[H]
	\centering
	\textbf{Hit Success Means}\\[4pt]
	\begin{adjustbox}{width=\columnwidth}
		\begin{tabular}{|c|c|c|c|c|}
			\hline
			\rowcolor[HTML]{9B9B9B} 
			& \textbf{Grab} & \textbf{Swipe} & \textbf{Throw} & \textbf{Tilt} \\ \hline
			Push & 95.9\% & 95.8\% & 92.6\% & 83.3\% \\ \hline
			Pull & 94\% & 97.5\% & 96.4\% & 71.5\% \\ \hline
		\end{tabular}
	\end{adjustbox}
	\caption{Success rate for each technique}
	\label{tab:successRate}
\end{table}

\subsection{Time taken}
This section will be presenting results based on the \target.
Here, we will be presenting the results in regards to how long each user took in performing each technique. 
When discussing these results, we will be referring to them as a techniques efficiency.

We performed a linear mixed effects model analysis on the data to see whether or not each technique had a significant effect on the efficiency of each attempt. 

\subsection{Distance from target}
The data that was used to measure the distance from the target was based on the \accuracy experiment.
Here we will present the results in regards to how far away from the center of the target (in pixels) each user was when performing the technique. 
This will be referred as a a techniques accuracy when we discuss the results presented in this section. 

We performed another linear mixed effects model analysis on the data to see if each technique had a significant effect on the accuracy of each attempt. 



