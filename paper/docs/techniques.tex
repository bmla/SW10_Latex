% !TEX root = ../paper.tex
\section{Interaction techniques} \label{sec:techniques}
In this section we will illustrate and describe the interaction techniques we developed in order to exchange data between mobile phones and large displays.
This is done so that we are capable of comparing them to one another in an emperical way.

There were several criteria behind the choice of these techniques. 
The main one was that users would be able to walk up to a large display and utilize them, which means that they should be as intuitive and natural as possible. 
\cref{tab:techniqueCriteria} shows the set of criteria that we based our choices of techniques on.

\begin{table}[H]
	\centering
	\begin{tabular}{|p{0.2\columnwidth}|p{0.7\columnwidth}|}
		\hline
		\rowcolor[HTML]{9B9B9B} 
		\textbf{Criteria} & \textbf{Description} \\ \hline
		Number of hands & There must be both one-handed and two-handed techniques. \\ \hline
		Previously used & To avoid designing and testing a set of novel techniques, we had the criterion that all techniques must have been used by others before we would use them. \\ \hline
		Complexity & The techniques must differ in their complexity and therefore we included techniques with different amount of steps. \\ \hline
		Natural feel & There must be a natural feel to the techniques in some way. \\ \hline
		Time & The time it takes to perform the different techniques must be different. \\ \hline
	\end{tabular}
	\caption{This table describes the set of criteria.}
	\label{tab:techniqueCriteria}
\end{table}
%Two-handed techniques which are techniques that require both hands.



All technique themes were found in the literature.
We then created a mirror version of each technique so they each would have a pull and push functionality.
