% !TEX root = ../paper.tex
\section{Interaction techniques} \label{sec:techniques}
In this section we will illustrate and describe the interaction techniques we developed in order to exchange data between mobile phones and large displays.
This is done so that we are capable of comparing them to one another in an empirical way.

There were several criteria behind the choice of these techniques. 
The main one was that users would be able to walk up to a large display and utilize them, which means that they should be as intuitive and natural as possible. 
\Cref{tab:techniqueCriteria} shows the set of criteria that we based our choices of techniques on.

\begin{table}[H]
	\centering
	\begin{tabular}{p{0.2\columnwidth} p{0.7\columnwidth}}
		\hline
		\textbf{Criteria} & \textbf{Description} \\ \hline
		\textit{Natural feel} & There must be a natural and intuitive feel to the techniques in some way. \\ \hline
		\textit{Number of hands} & There must be both one-handed and two-handed techniques. \\ \hline
		\textit{Previously used} & To avoid designing and testing a set of novel techniques, we had the criterion that all techniques must have been used by others before we would use them. \\ \hline
		\textit{Complexity} & The techniques must differ in their complexity and therefore we included techniques with different amount of steps. \\ \hline
		\textit{Activation method} & The way each technique is activated must be different from each other. \\ \hline
	\end{tabular}
	\caption{Criteria}
	\label{tab:techniqueCriteria}
\end{table}
%Two-handed techniques which are techniques that require both hands.


All technique themes were found in the literature.
We then created a mirror version of each technique so they each would have a \push and \pull version.
\push means that the user will be pushing information from the mobile to the large display, and \pull means the oposite, the user will pull information from the screen onto the mobile device. 

Eight techniques were chosen in the end: \alltechniques, each with a \push and \pull variant

The \grab technique is based on a grabbing gesture presented by Hespanhol et al. \cite{Hespanhol:2012} as one of five proposed gestures.
A related technique is described by Markussen et al. \cite{} and presents a mid-air word-gesture keyboard named \emph{``Vulture''} that uses a pinch gesture (touching index finger to thumb) to give the user control of the pointer which can then be used to select letters.
A variation of the \grab technique (\cref{fig:grabTechnique}) is used in Memory Stones \cite{Ikematsu:2015} by Ikemasu et al. as part of an system for exchanging information between different devices. Benko and Wilson \cite{Benko:2010} used the \grab technique in a system were the user would interact with visualizations inside a dome. \grab is a combination of the grabbing gesture and the pointing technique used by Scheible et al. \cite{Scheible:2008}.
This technique was chosen because we wanted to simulate the feeling of picking up an object of interest and placing it on a desired location.
\grab is a complex technique, requiring a series of steps as well as using both hands to complete the interaction.
The \push version of this technique is completed as follows: the user first grab a object of interest from the telephone by pinching it with his fingers (\cref{fig:grabTechniqueA}), closing his hand, and metaphorically putting the object in his hands.
The user then raises his closed hand and aims at the screen where he wants to place the object (\cref{fig:grabTechniqueB}).
The final step is to open the hand over the desired placement of the object on the large display (\cref{fig:grabTechniqueC}).
The \pull version is a bit different.
The user first places his open hand over the object of interest on the large display (\cref{fig:grabTechniqueC}).
The user then closes hand over the object (\cref{fig:grabTechniqueB}) and finally places it on his phone by touching it with his closed hand (\cref{fig:grabTechniqueA}).  

\todo[inline]{pictures have to be made anew. they dont fit the new push pull idea}

\begin{figure}[H]
	\subfloat[]{\includegraphics[width = 0.33\columnwidth]{images/grab_a.jpg}\label{fig:grabTechniqueA}}
	\subfloat[]{\includegraphics[width = 0.33\columnwidth]{images/grab_b.jpg}\label{fig:grabTechniqueB}}
	\subfloat[]{\includegraphics[width = 0.33\columnwidth]{images/grab_c.jpg}\label{fig:grabTechniqueC}}
	\caption{Grab technique}
	\label{fig:grabTechnique}
\end{figure}

The \swipe technique (\cref{fig:swipeTechnique}) was utilized by Bragdon et. al in Code Space \cite{Bragdon:2011}.
He developed a system that would support developer meetings with the help of smart phones and the Kinect. Bragdon describes the technique as \emph{``cross-device interaction with touch and air pointing''} and the swiping motion is described as \emph{``flicking up on the touch screen''}.
This technique was chosen because it has a very simplistic design, with a very low level of complexity since it requires very few steps to activate.
It is also a one handed technique and requires very little effort from the user to use.
The \push and \pull version of this technique are very similar.
First the user points at the desired location with the phone in a stretched arm (\cref{fig:swipeTechniqueA}) and then swipes his finger on the screen (\cref{fig:swipeTechniqueC}).
The direction he swipes depends on whether the user wants to \push or \pull information.
If he swipes away from himself, he pushes information to the screen.
If he swipes towards himself, he is pulling information from the screen onto the device.  

\begin{figure}[H]
	\subfloat[]{\includegraphics[width = 0.33\columnwidth]{images/swipe_a.jpg}\label{fig:swipeTechniqueA}}
	\subfloat[]{\includegraphics[width = 0.33\columnwidth]{images/swipe_b.jpg}\label{fig:swipeTechniqueB}}
	\subfloat[]{\includegraphics[width = 0.33\columnwidth]{images/swipe_c.jpg}\label{fig:swipeTechniqueC}}
	\caption{Swipe technique}
	\label{fig:swipeTechnique}
\end{figure}

The \throw technique (\cref{fig:throwTechnique}) is a combination of two techniques.
The first is a pointing technique used by Scheible et al. in MobiToss \cite{Scheible:2008} were the user uses his hands as a cursor on the screen .
The second is a technique used by Walter et al. in Cuenesics \cite{Walter:2014} were the system is used to submit information onto a large public display.
This technique was chosen because of it's natural and playful design, as well as mirroring the idea of throwing something, like a ball, to someone or somewhere.
\throw is a two handed technique, as well as having a larger number of steps to take in order to activate it.
The \throw technique is performed as follows: 
First the user points at the targeted position with one hand (\cref{fig:throwTechniqueA}).
Then, if the user is pushing data from the phone, he has to select the data to be pushed (\cref{fig:throwTechniqueB}).
The user then performs a swinging motion with the other hand which is holding the phone.
If he is pushing, the he swings towards the screen, if he is pulling he swings away from it (\cref{fig:throwTechniqueC}) 


\begin{figure}[H]
	\subfloat[]{\includegraphics[width = 0.33\columnwidth]{images/throw_a.jpg}\label{fig:throwTechniqueA}}
	\subfloat[]{\includegraphics[width = 0.33\columnwidth]{images/throw_b.jpg}\label{fig:throwTechniqueB}}
	\subfloat[]{\includegraphics[width = 0.33\columnwidth]{images/throw_c.jpg}\label{fig:throwTechniqueC}}
	\caption{Throw technique}
	\label{fig:throwTechnique}
\end{figure}

The \tilt technique (\cref{fig:tiltTechnique}) is used by Lucero et al. in MobiComics \cite{Lucero:2012}.
He created a system in which users would transfer objects from a large display onto their mobile devices.
The \throw technique chosen because it is a one handed, low complexity technique with a few amount of steps needed to activate it. 
Just like the \swipe technique, it is easy intuitive and straight forward. 
The \tilt technique is performed as follows:
The user first points and the target location (\cref{fig:tiltTechniqueA}).
If the user is performing a \push technique, he tilts the phone away from himself.
If he is performing a \pull technique, he tilts the phone towards himself. 

\begin{figure}[H]
	\subfloat[]{\includegraphics[width = 0.33\columnwidth]{images/tilt_a.jpg}\label{fig:tiltTechniqueA}}
	\subfloat[]{\includegraphics[width = 0.33\columnwidth]{images/tilt_b.jpg}\label{fig:tiltTechniqueB}}
	\subfloat[]{\includegraphics[width = 0.33\columnwidth]{images/tilt_c.jpg}\label{fig:tiltTechniqueC}}
	\caption{Tilt technique}
	\label{fig:tiltTechnique}
\end{figure}

All techniques that were chosen have all been used before in other systems in order to facilitate the interaction between mobile devices and large public displays.
They are all different in the way they are activated as well as the number of hands that are used to perform the technique.
For our two handed techniques, we have \throw and \grab, and for our one handed techniques we have \tilt and \swipe. 
Each technique also has different ways to activate it. 
The \throw and \tilt techniques require the user to move the phone, whereas the \swipe and \grab techniques require the user to touch and manipulate the screen of the device in order to activate. 

All the techniques that will be used in this experiment use some combination of mid-air pointing, touch gestures and phone movements to perform. 
These were all achieved by using the Microsoft Kinect V2 as well as the accelerometer and touch sensors on the phones. 
The Kinect utilizes its depth camera in order to give information about a users location in 3D space, allowing us to track the position of the users hands and building the techniques around that. 
The touch sensor on the phones was used in order to recognize touch and swipe gestures; the accelerometer was used to detect significant movement on the phone and use that for the \tilt and \throw techniques. 