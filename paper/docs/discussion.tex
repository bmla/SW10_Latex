% !TEX root = ../paper.tex
\section{Discussion}\label{sec:discussion}
When discussing the results from our experiment, it is important to make sure that all terms used are clearly defined. When we talk about \emph{Effectiveness}, we refer to a technique's success rate, or how often it successfully hit the target on the screen. When talking about \emph{Efficiency}, we refer to how fast a technique is to perform. Finally, when we talk about \emph{Accuracy}, we are talking about how precise the technique is, or how close to the center of the target the attempt for the technique was. 

\subsection{Effectiveness}

Our results for effectiveness, or the success rate of each technique, show that there is in fact some association between each technique and whether or not an attempt was successful.
Success is a very simple term and as such a great deal of information about the success of a technique is lost when the answer is a simple hit or miss. 
In order to better understand the association between a technique and a successful attempt, we can use accuracy as a more specific and precise definition of success.
The closer an attempt is from the center of the target, the more successful it is.
We will therefore use accuracy as more precise measure of a techniques success (see \cref{sec:accuracy}).


\subsection{Efficiency}

Once we start looking at the results regarding the efficiency of each technique, some rather interesting things come to light.

We see that \swipe is the fastest technique by far.
With an mean time per attempt of 3.78 seconds, it outperforms the next fastest technique, \throw, with a mean time of 4.39 seconds per attempt. 
One might have expected both one handed techniques to be the fastest, since they have very few steps that must be taken to activate these techniques, but \throw is faster than \tilt. 
This can be due to a fact that people felt very uncomfortable with the \tilt technique.
When performing it, users would be very cautious because the cursor tended to move a large distance after each attempt.
This would lead to people performing a very cautious and slow tilting movement with the phone, and thus not triggering the technique and having to perform it multiple times. 
Not surprisingly \grab is the slowest technique, since this is by far the most complex technique, with the most number of steps needed in order to perform it.

Another interesting thing to notice is that a successful attempt did not, in fact, have a significant effect on the time for each attempt.
This makes sense when you consider it carefully: the aiming part of each technique is shared between all techniques and as such should take the same amount of time.
Once the user has acquired the target, he starts performing the technique, and this is where the difference in time comes from.

As such, it is not surprising to see that the target size does have a significant effect on time.
The aiming process for the smaller targets takes longer, and affects all techniques similarly.

We also note that the direction of each attempt did not have a significant effect on the efficiency of each technique. 
This is because not all techniques are effected similarly by the direction. 
While \swipe, \throw and \tilt all take less time to perform while pulling, \grab has the opposite effect.
This can be explained once you examine the interaction between direction, technique and success.
Our pairwise comparisons show that the significant interaction between direction, technique and success comes the from difference between the successful and unsuccessful attempts for the \grab \pull technique. 
Here it shows that an unsuccessful attempt with the \grab \pull is significantly slower than a successful one. 
This can be explained by the implementation of the technique. 
Once the user closes his hand in an attempt to grab the shape on the screen, he is no longer allowed to retry, even if he missed.
This is counter-intuitive to reality, where if someone was to miss a object he was grabbing, he would simply open his hand and retry.
This confused our participants. 
Whenever they missed, as they would try to open their hand in order to get another attempt at grabbing the objects, but the system would not allow that.
As such, the \grab \pull technique did not live up to the metaphor of grabbing and releasing objects in real life.
This was an intentional design decision though, since none of the other techniques had the opportunity  of retrying the attempt.
If we had allowed \grab \pull to do so, we would most likely have ended up with a perfect hit rate for that technique. 

\subsection{Accuracy}
\label{sec:accuracy}
We see that out of the four techniques, \swipe was the most accurate technique, having a mean of 13.27 pixels from the center of each target. 
This is closely followed by \throw, with 15.65 pixels, and \grab, with 17.22 pixels.
The \tilt technique trails far behind, with a mean distance of 30.31 pixels from the center of each target.   
It is no surprise that \tilt is so far behind, since it requires the user to do a very subtle movement with the same hand they are pointing with.
This usually leads to the cursor being moved away from its original placement since it is very hard to adjust the placement of the hand in such a way as to take into account the required activation movement. 
Several users attempted to place the cursor slightly below the target to compensate for the movement. 

Even though we asked our participants to be as precise as possible in the \accuracy and aim for the white cross in the middle of the target, we see that the target size actually does have an effect.
This is likely related to the idea that the shape itself is still the target. 
Once the users aim as close as possible to the center of the shape, they prepare themselves to perform the gesture. 
The cursor also starts to deviate from the center since it is almost impossible to hold it completely still for the duration of the technique.
Since it does not take much movement before the cursor leaves the small target's shape area, users must constantly realign the cursor with the center of the shape before performing the technique.
This is not the case with the larger targets, where the cursor is allowed to deviate more from the center before it actually leaves the shape, and as such not prompting the need to realign the cursor towards the center, since it is still inside the shape. 
This suggests that users would initially aim towards the center of the cross and be as precise as possible. 
As they were to perform the gesture, as long as the cursor was still inside the shape, they would go ahead and perform the technique.

\subsection{Summary}
Initially, we had divided the techniques by assigning them different attributes. 
This is because we believed these attributes would heavily effect the results of these techniques compared to each other.
We gave each technique two attributes, the amount of hands needed to perform the technique, and whether or not the phone was in movement during the activation of the technique.
\Cref{tab:division} shows the division that we made.

\begin{table}[H]
	\centering
	
	\def\arraystretch{1.8}
	\begin{tabular}{c c c}
		& \textbf{Touch phone} & \textbf{Move phone} \vspace{1mm} \\ \hline
		\textbf{One handed} & \startCirc{Swipe} & Tilt \\  \hline
		\textbf{Two handed} & Grab & \endCirc{Throw} \\  \hline
	\end{tabular}
	\vspace{2mm}
	\caption{Attribute division for techniques, with the most efficient and accurate techniques highlighted}
	\label{tab:division}
\end{table}

The data shows that \swipe is the most efficient and accurate technique out of all the four techniques.
We then see that \throw follows closely behind, in both efficiency and accuracy. 
If we look at efficiency, we see that \tilt takes considerably less time to perform than \grab.
In regards to accuracy, \grab is the third most accurate technique, with \tilt far behind as the least accurate technique.

While \swipe and \throw do not share any of the attributes we assigned them, they do have one thing in common: the cursor pointing hand is held still through out the activation process.
The \tilt technique requires the phone hand, which is also the cursor pointing hand, to be moved in order to activate, and \grab requires the user to open or close his cursor pointing hand, which causes a slight movement of the cursor when doing so.
This seams to be the single largest contributing factor in whether or not the technique was efficient and accurate.
Our findings suggest that the most accurate and efficient techniques are the ones were it is possible to hold the cursor still while performing the technique.

We then see that \grab, a two handed technique, is significantly more accurate than \tilt.
This suggests that being capable of operating the mobile device in one hand and aiming with the other contributes to the stability of the cursor, making two hand gestures more accurate than one handed.
This is also suggested by the findings in \cite{Seifert:2013}.
You also see that \tilt is significantly faster than \grab.
This suggests that if cursor stability was not a problem, then one handed techniques might be faster to perform than two handed.
In order to properly determine whether or not the amount of hands used to perform a technique has any effect on efficiency and accuracy, further research needs to be conducted. 


