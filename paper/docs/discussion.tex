% !TEX root = ../paper.tex
\section{Discussion}\label{sec:discussion}
When discussing the results we uncovered from our experiment, it's important to make sure that all terms used are clearly defined. When we talk about \emph{Effectiveness}, we talk about a techniques success rate, or how often it successfully hit the target on the screen. When talking about \emph{Efficiency}, we are referring to how fast a technique is to perform. Finally, when we talk about \emph{Accuracy}, we are talking about how precise the technique is, or how close to the center of the target the attempts for the technique was. 

\subsection{Effectiveness}

When looking at the results for the \push techniques, \grab has the highest effectiveness of all the techniques, followed by \swipe, then \throw and finally \tilt. It is interesting to note that the \grab and \swipe are incredibly close to each other, with \throw trailing closely behind. 

If we look at the \pull techniques, we see \swipe having the highest success rate, followed by \throw and finally \grab. It is a very similar image to the \push techniques, having these three techniques lay very close to each other with they success rate, and \tilt being rather far behind.  

It is interesting to note that while \grab, \swipe and \throw are almost equally effective, \tilt is rather far behind. This is also the technique that requires the user to operate it with one hand, as well as move the phone with one hand in order to activate the technique. The technique requires the user to perform a rather subtle movement with the same hand he is using to point at the screen with. This clearly effects the precision of the technique, as the pointer would almost always be dislocated from the location the user intended to hit due to the activation movement.

