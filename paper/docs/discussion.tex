% !TEX root = ../paper.tex
\section{Discussion}\label{sec:discussion}
When discussing the results from our experiment, it is important to make sure that all terms used are clearly defined. When we talk about \emph{Effectiveness}, we refer to a technique's success rate, or how often it successfully hit the target on the screen. When talking about \emph{Efficiency}, we refer to how fast a technique is to perform. Finally, when we talk about \emph{Accuracy}, we are talking about how precise the technique is, or how close to the center of the target the attempt for the technique was. 

\subsection{Effectiveness}

When looking at the results for the \push techniques, all techniques, with the exception of \throw, had a significant association with the success rate of each attempt.
The \push techniques had an expected success rate of 92.02\%, which \grab (95.9\%) and \swipe (96\%) clearly surpassed.
The \tilt technique had a much lower success rate(83.3\%), while \throw did not have a significant association with the success rate.

If we look at the results for the \pull techniques, which had an expected success rate of 89.92\%, all of them had a significant association with the success rate of each attempt.  
The \swipe technique was clearly the most effective technique, with a success rate of 97.5\%, followed closely by \throw (96.4\%) and \grab (94\%). 
The \tilt technique falls far behind with a success rate of only 71.5\%. 

It is unclear why the \push version of the \throw technique does not have a significant association with success.
This could possible be because the pulling motion of the \throw technique was less familiar to participants and as such lead to a more careful, precise motion.
This effect can also be seen when we discuss accuracy, where users were explicitly asked to be more precise.
This lead to the \push and \pull version being much more closely related, since users were performing more careful and precise motions in both directions. 

In order to properly determine this, further experiments must be conducted.
We can see though that \swipe and \grab are both very effective techniques, both while pushing and pulling information from a display.



\subsection{Efficiency}

Once we start looking at the results regarding the efficiency of each technique, some rather interesting things come to light.

We see that \swipe is the fastest technique by far.
With an mean time pr. attempt of 3.78 seconds, it outperforms the next fastest technique, \throw, at a mean time of 4.39 seconds per attempt. 
One might have expected both one handed techniques to be the fastest, since they have very few steps that must be taken to activate these techniques, but \throw is faster than \tilt. 
This can be due to a fact that people felt very uncomfortable with the \tilt technique.
When performing it, users would be very cautious because the pointer tended to move a large distance after each attempt.
This would lead to people performing a very cautious and slow tilting movement with the phone, and thus not triggering the technique and having to perform it multiple times. 
Not surprisingly \grab is the slowest technique, since this is by far the most complex technique, with the most number of steps needed in order to perform it.

Another interesting thing to notice is that a successful attempt did not, in fact, have a significant effect on the time for each attempt.
This makes sense when you consider it carefully: the aiming part of each technique is shared between all techniques and as such should take the same amount of time.
Once the user has acquired the target, he starts performing the technique, and this is where the difference in time comes from.

As such, it is not surprising to see that the target size does have a significant effect on time.
The aiming process for the smaller targets takes longer, and affects all techniques similarly.

We also note that the direction of each attempt did not have a significant effect on the efficiency of each technique. 
This is because not all techniques are effected similarly by the direction. 
While \swipe, \throw and \tilt all take less time to perform while pulling, \grab has the opposite effect.
This can be explained once you examine the interaction between direction, technique and success.

Our pairwise comparisons show that the significant interaction between direction, technique and success comes the from difference between the successful and unsuccessful attempts for the \grab \pull technique. 
Here it shows that an unsuccessful attempt with the \grab \pull is significantly slower than a successful one. 
This can be explained by the implementation of the technique. 
Once the user closes his hand in an attempt to grab the shape on the screen, he is no longer allowed to retry, even if he missed.
This is counter-intuitive to reality, were if someone was to miss a object he was grabbing, he would simply open his hand and retry.
This confused our participants. 
Whenever they missed, as they would try to open their hand in order to get another attempt at grabbing the objects, but the system would not allow that.
As such, the \grab \pull technique did not live up to the metaphor of grabbing and releasing objects in real life.
This was an intentional design decision though, since none of the other techniques had the opportunity  of retrying the attempt.
If we had allowed \grab \pull to do so, we would most likely have ended up with a 100\% precise technique. 

\subsection{Accuracy}
We see that out of the four techniques, \swipe was the most accurate technique, having a mean of 13.27 pixels from the center of each target. 
This is closely followed by \throw, with 15.65 pixels, and \grab, with 17.22 pixels.
The \tilt technique trails far behind, with a mean distance of 30.31 pixels from the center of each target.   
It is no surprise that \tilt is so far behind, since it requires the user to do a very subtle movement with the same hand they are pointing with.
This usually leads to the pointer being moved away from its original placement since it is very hard to adjust the placement of the hand in such a way as to take into account the required activation movement. 
Several users attempted to place the cursor slightly below the target to compensate for the movement. 

It is again interesting to note that direction has no effect on the accuracy of the attempt. 
This points to the symmetry of our techniques, and that both the \pull and the \push versions of each technique share very similar pointing methods.

Even though we asked our participants to be as precise as possible in the \accuracy and aim for the white cross in the middle of the target, we see that the target size actually does have an effect.
This is likely related to the idea that the shape itself is still the target. 
Once the users aim as close as possible to the center of the shape, they prepare themselves to perform the gesture. 
The cursor also starts to deviate from the center since it is almost impossible to hold it completely still for the duration of the technique.
Since it does not take much movement before the cursor leaves the target area, users must constantly realign the cursor with the center of the shape before performing the technique.
This is not the case with the larger targets, where the cursor is allowed to deviate more from the center before it actually leaves the shape, and as such not prompting the need to realign the cursor towards the center, since it is still inside the shape. 
