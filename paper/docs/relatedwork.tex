% !TEX root = ../paper.tex
\section{Related work} \label{sec:relatedwork}
In this section we will be present the work which has been done in the area of transferring information and data between phones and displays.
We will present the developments within the topic of pointing in mid-air and that of controlling a cursor on a display at a distance.

\subsection{Handheld and Large Display Interaction techniques} \label{sec:techniquesForTransfer}
Techniques for interacting with large displays using a handheld device such as a smartphone are numerous.
Different approaches have been documented in the literature and amongst them are \todo{...text goes here...}
% \todo[inline]{insert first related work reference here}

\subsection{Mid-air pointing} \label{sec:midAirPointing}
For pointing in mid-air there are different applications such as Microsofts's new mixed reality glasses, named HoloLens.
With HoloLens, the controlling interface is hand gestures combining the physical 3D world with the virtual or augmented reality made possible with the HoloLens.
Using mid-air gestures could also make the 3D space we move around in combine more seamlessly with the what we see on a large display.
In the literature there are different approaches to pointing and controlling for example virtual pointers on a screen. 
%One of them are \todo[inline]{insert first related work reference here}

% This one is about pointing techniques
Nancel et al. \cite{Nancel:2013} focuses on high precision pointing techniques for acquiring targets on a large wall sized display.
Their implementation uses the handheld device for controlling the pointer on the large display and a small area of the handheld device is used for relative pointing.
One technique uses two fingers for coarse pointing and one finger for precision pointing. 
Another technique they use is a head-based coarse pointing technique making it possible for the participants to roughly get the pointer close to the target using their head.
The two techniques both have a discrete and a continuous mode giving a total of 4 techniques that  were compared with each other but also compared to state-of-the-art techniques.
The results of their experiments show that continuous head pointing is faster and more successful than their other techniques.
A comparison showed that their technique performed as good as some of the state-of-the-art techniques and that it was in fact possible to maintain precision and screen real estate on the handheld device.