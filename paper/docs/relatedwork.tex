% !TEX root = ../paper.tex
\section{Related work} \label{sec:relatedwork}
In this section we will be present the work which has been done in the area of transferring information and data between phones and displays.
We will present the developments within the topic of pointing in mid-air and that of controlling a cursor on a display at a distance.

\subsection{Large displays \& Mid-air pointing} \label{sec:largeDisplayAirPointing}
In this section we will discuss the research within the area of large displays and wall-sized displays.
The interaction interface for large displays will most likely be either touch or mid-air pointing.
For pointing in mid-air there are different applications such as Microsofts's new mixed reality glasses, named HoloLens.
With HoloLens, the controlling interface is hand gestures combining the physical 3D world with the virtual or augmented reality made possible with the HoloLens.
Using mid-air gestures could also make the 3D space we move around in combine more seamlessly with the what we see on a large display.
In the literature there are different approaches to pointing and controlling, for example, virtual pointers on a screen. 

% This is the Off-Limits paper
Using a large display and mid-air pointing, Markussen et al. \cite{Markussen:2016} explored an interaction concept called \emph{Off-Limits} in which the user is able to interact with a large display outside the boundaries of the screen.
They did three studies and in the first two they were exploring the performance and how people are understanding off-screen space.
For the last experiment they compared the \emph{Off-Limits} interface (which was created based on knowledge from the two previous studies) with the naïve implementation of \emph{Off-Screen} pointing.
Participants had to acquire a number by pointing on a horizontal line that continued beyond the screen's boundary in both directions.
The results for the last study showed that \emph{Off-Limits} outperformed the \emph{Off-Screen} technique for time when the target values were far away from the screen.
For the number of interaction a participant had to do they found that \emph{Off-Limits} require less interactions than \emph{Off-Screen} and that all except the two closest values to the screen were significant.

% Should I Stay or Should I Go
Jacobsen et al. explores two different interaction interfaces for large displays, namely touch and mid-air gestures

% CodeSpace

\subsubsection{Using handheld devices} \label{sec:midAirPointingHandheld}
We will explore the area of pointing in mid-air with a handheld device while using the handheld device, for example to interact with a large display.

% This one is about pointing techniques where they are using handdeld devices (both smartphones and tablests)
Nancel et al. \cite{Nancel:2013} focuses on high precision pointing techniques for acquiring targets on a large wall sized display.
Their implementation uses the handheld device for controlling the pointer on the large display and a small area of the handheld device is used for relative pointing.
One technique uses two fingers for coarse pointing and one finger for precision pointing. 
Another technique they use is a head-based coarse pointing technique making it possible for the participants to roughly get the pointer close to the target using their head.
The two techniques both have a discrete and a continuous mode giving a total of 4 techniques that  were compared with each other but also compared to state-of-the-art techniques.
The results of their experiments show that continuous head pointing is faster and more successful than their other techniques.
A comparison showed that their technique performed as good as some of the state-of-the-art techniques and that it was in fact possible to maintain precision and screen real estate on the handheld device.

\subsection{Techniques for transfering data \& Target acquisition} \label{sec:targetAcquisition}
In \emph{``Proximal and Distal Selection of Widgets''} Rashid et al. \cite{Rashid:2011} explores two different techniques for interacting with and acquiring targets on a large display using a handheld device.
The first technique is \emph{Proximal Selection (PS)} which pulls a selected, or zoomed-in, area of the large display onto the phone and the user will then be able to select the correct target.
The other technique is \emph{Distal Selection (DS)} where the user will point at the large display, zoom-in on the selected area, and finally select the desired target on the large display.
In the experiment they found that, for complex tasks and with regards to time, \emph{PS} outperforms \emph{DS} but for simpler tasks \emph{DS} was approximately 0.1 sec faster and the effect was insignificant.
The error rate for the techniques showed that \emph{DS} had fewer missed clicks for both small and large targets and that there was a significant difference for the small targets.

Techniques for interacting with large displays using a handheld device such as a smartphone are numerous.
Different approaches have been documented in the literature and amongst them are throw and tilt gestures for interacting with large displays.
Dachselt et al. \cite{Dachselt:2008} and Boring et al. \cite{Boring:2009} describes how a tilt technique with a handheld device can be used to control a pointer on a remote display.
In addition, Dachselt et al. describes a throwing gesture for transferring data (e.g. from a phone) to and from a large display and the application proposed also describe the concept of transferring an entire user interface between a phone and a display using the throw gesture.